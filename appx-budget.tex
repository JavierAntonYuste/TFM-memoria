\chapter{Budget}
\label{sec:budget}


The budget elaborated for this project consists of the following items.
\begin{itemize}
    \item Material execution budget
        \begin{itemize}
            \item Material resources budget
            \item Labour costs
        \end{itemize}
    \item Total project costs
    \item Overheads and industrial profit
    \item Total budget
\end{itemize}

\section*{Material execution budget}
The material execution budget together with the overheads and the industrial profit constitute what is called the contract execution budget, which is referred to as the total budget.

All budgets shall be presented in Euros (\euro).

\subsection*{Material resources budget}

Next, in the Table~\ref{tab:hardware}, we find the material resources used throughout the project.

\begin{table}[]
\centering
\begin{tabular}{|lrr|r|}
\hline
\multicolumn{1}{|c|}{\textbf{Concept}} & \multicolumn{1}{c|}{\textbf{Cost}} & \multicolumn{1}{c|}{\textbf{Amortisation}} & \multicolumn{1}{c|}{\textbf{Real cost}} \\ \hline
\multicolumn{1}{|l|}{Principal laptop} & \multicolumn{1}{r|}{1.945,10}      & 16.66\%                                    & 324.05                                  \\ \hline
\multicolumn{1}{|l|}{Server PC}        & \multicolumn{1}{r|}{820,00}        & 16.66\%                                    & 136,37                                  \\ \hline
\multicolumn{1}{|l|}{Screen}           & \multicolumn{1}{r|}{531,19}        & 16.66\%                                    & 88.5                                    \\ \hline
\multicolumn{1}{|l|}{Keyboard}         & \multicolumn{1}{r|}{15}            & 16.66\%                                    & 2.5                                     \\ \hline
\multicolumn{1}{|l|}{Mouse}            & \multicolumn{1}{r|}{15}            & 16.66\%                                    & 2.5                                     \\ \hline\hline
\multicolumn{3}{|l|}{\textbf{Total cost}}                                                                                         & \textbf{553.92}                         \\ \hline
\end{tabular}
\caption{Material resources budget}
\label{tab:hardware}
\end{table}

\subsection*{Labour costs}
The project required the participation of an engineer who implemented the system and put it into operation. This profile falls within contribution group 1 of the General Social Security Scheme, with a working day of 8 hours a day and 21 days a month. The complete breakdown can be consulted in the Table~\ref{tab:labour}.

\begin{table}[h]
\centering
\begin{tabular}{|lcc|r|}
\hline
\multicolumn{1}{|c|}{\textbf{Position}} & \multicolumn{1}{c|}{\textbf{Hours}} & \textbf{Price/Hour}     & \multicolumn{1}{c|}{\textbf{Total}} \\ \hline
\multicolumn{1}{|l|}{Engineer}          & \multicolumn{1}{r|}{1050}           & \multicolumn{1}{r|}{16} & 16800                               \\ \hline\hline
\multicolumn{3}{|l|}{\textbf{Total cost}}                                                                        & \textbf{16800}                      \\ \hline
\end{tabular}
\caption{Labour Costs}
\label{tab:labour}
\end{table}


\subsection*{Total resources cost}
The total cost of resources is made up of the cost of material resources and labour. Its breakdown can be seen in the Table~\ref{tab:totalresources}.

\begin{table}[]
\centering
\begin{tabular}{|l|r|}
\hline
\multicolumn{1}{|c|}{\textbf{Concept}} & \multicolumn{1}{c|}{\textbf{Cost}} \\ \hline
Material resources                     & 16800                              \\ \hline
Labor                                  & 553.92                             \\ \hline\hline
\textbf{Total cost}                    & \textbf{17353.92}                  \\ \hline
\end{tabular}
\caption{Total resources costs}
\label{tab:totalresources}
\end{table}

\section*{Overheads and industrial profit}
Under the chapter of general expenses are included all those indirect expenses derived from the use of installations, depreciations, fiscal expenses, etc. With this, the execution budget by contract is as shown in the Table~\ref{tab:overheads}.

\begin{table}[h]
\centering
\begin{tabular}{|l|r|}
\hline
\multicolumn{1}{|c|}{\textbf{Concept}}        & \multicolumn{1}{c|}{\textbf{Cost}} \\ \hline
Material Execution Budget                     & 17353.92                           \\ \hline
Overheads (16\% of MEB)                       & 2776.63                            \\ \hline
Industrial profit (6\% of MEB)                & \multicolumn{1}{l|}{1041.23}       \\ \hline\hline
\textbf{Total contract implementation budget} & \textbf{21171.78}                  \\ \hline
\end{tabular}
\caption{Overheads and industrial profit}
\label{tab:overheads}
\end{table}

\section*{Total Budget}
Applying 21\% Value Added Tax (VAT), the total budget shown in the table below is obtained. the Table~\ref{tab:total}. The total budget of the project amounts twenty-five thousand six hundred and seventeen and eighty-five cents.

\begin{table}[h]
\centering
\begin{tabular}{|l|r|}
\hline
\multicolumn{1}{|c|}{\textbf{Concept}} & \multicolumn{1}{c|}{\textbf{Cost}} \\ \hline
Budget subtotal                        & 21171.78                           \\ \hline
VAT (21\%)                             & 4446.07                            \\ \hline
\textbf{Total budget}                  & \textbf{25617.85}                  \\ \hline
\end{tabular}
\caption{Total budget}
\label{tab:total}
\end{table}

\begin{flushright}
  Madrid, 23rd June 2022\\
  \rule{0pt}{2cm}

  Javier Antón Yuste\\
\end{flushright}