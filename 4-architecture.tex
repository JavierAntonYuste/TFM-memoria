\chapter{System architecture}
% arquitectura del sistema
%API python
% web
\label{chap:architecture}

\textit{In this chapter the system architecture will be shown. We will explain how we have implemented the web server, the user interface and the automations created for it.}

\clearpage

\section{General Architecture}

%This chapter will serve as a description of the methodology that has been followed in order to develop the system and meet the requirements set out in the Chapter~\ref{chap:introduction}. As mentioned, this work arises from the need of tackling a problem in today's society, namely Eating Disorders. More specifically, the aim is to help in the early detection of such illnesses so that they can be treated as soon as possible.

The aim of this project is to help in the early detection of Eating Disorders so that they can be treated as soon as possible. To achieve that, the system's presentation should be easy to handle and understandable for the people who use it. Furthermore, it has to be visually appealing in order to motivate people with a potential Eating Disorder to use it. It is desired to include the option of performing some actions after the prediction is taken as well.

The system architecture that has been developed for this Master Thesis can be seen in the Figure~\ref{fig:architecture}, where it can be found the interactions between the subcomponents in a simplified diagram.

% Mostrar esquema arquitectura
\begin{figure}[h]
    \centering
    \includegraphics[width=1.05\textwidth]{img/architecture/architecture.png}
    \caption{Simplified diagram of the architecture of the system.}
    \label{fig:architecture}
\end{figure}


In this figure we can see a simplified abstract diagram of the structure of our system, consisting of two distinct main elements, which are the server made with Flask and the user interface for which Flutter has been used. In the following this architecture will be broken down and zoomed in on the different parts to explain what the elements represented in the diagram above consist of and what function they have.

The presentation layer of this Master Thesis has been designed using a specific methodology based on states called BLoC, which will be explained in this chapter. There are diferent parts that are part of this BLoC architecture that need to be implemented in order to perform different actions that will enhance the user experience.

To serve this presentation layer with data, a server has been implemented in which an API has been constructed. The objective of this is to consult the information about the required user and predict the suffering of an Eating Disorder. This server makes calls to the Twitter API to make the prediction using the Machine Learning model previously preloaded on the same server. The model that has been followed for implementing this part of the system is called Front Controller, which will be explained here.

Besides, the system has been integrated with an Automation Platform powered by EWE. It includes semantic automation to project's system for helping to include different actions that users can take after the prediction is performed. This enhances the features of the platform and as a consequence, the user experience. We will talk about this topic in a separated Section in this Chapter.

\section{Flutter application}
% Presentacion de la aplicacion
The application aims to have a clear and guided user experience that engages the user for using it. It should have a clear flow for easing its manipulation and really help the person who uses it. Because of that we used Flutter for implementing it.

%Multiplataforma
It is a Flutter-based as it is mentioned, so that implies that it is a cross-platform application that can be run in Android, iOS and web, which make it polivalent. Its content is dinamically treated so it adjusts to all kinds and sizes of screen, which has been also take into account in the design process.

% Diseño de la app: BLOC
The design pattern followed is called BloC, which makes the application to be based on separate components. These components contain only the business logic, \textit{i.e.} the logic that determines how information can be created, stored and changed. BLoC is widely used in applications developed in Flutter, as it is based on widgets and this pattern provides the support to communicate these widgets with other components of the application.

\begin{figure}[!htp]
    \centering
    \includegraphics[width=0.8\textwidth]{img/architecture/bloc.png}
    \caption{BLoC architecture.}
    \label{fig:bloc}
\end{figure}

This communication is done through the issuing of events from the UI as shown in the Figure~\ref{fig:bloc}. BLoC components receive this events and take the actions that are needed, changing their state. Afterwards, they emit back stream back to the UI to make the relevant changes to the interface. 

In turn, the project's BLoC-based application is connected with repositories, which serve to obtain data necessary for the application, as in this case is the prediction of the system issued by the Flask server explained below. The repository of the system provides access to long-term state data, in our case via API.

Furthermore, this user interface was connected with the Task Automation Server. When a positive prediction is done, the webserver will trigger an event against the automation platform and the response of this platform will be received in our Flutter application. With this data some actions are displayed in user screen in order to help to advance in the diagnosis process.
\section{Flask server}
\label{sec:server}
% Funciones que realiza: Extraccion datos, procesamiento y predicción
 In this section focus will be put on the part that obtains and processes the data from the application in order to be able to make a prediction about the user that is analysed. The basic functions of this  web server are data extraction, data processing and the generation of a prediction which is returned in JSON format as a response, along with additional information.
 
 A technology that was fast and light was needed, as the project was designed to simplify the processes like the text processing. It is desired to be easy to query, in order to get the best user experience and reduce waiting times. For this reason Flask was chosen, a fairly mature lightweight framework for Python, which allows the project to set up a server in a matter of seconds and process requests quickly.
 
 
 
 % MODELO SEGUIDO en  arquitectura: FRONT CONTROLLER
 The architecture pattern followed to implement this Flask server is the Front Controller model, which can be seen in the Figure~\ref{fig:front-controller}. This model is based on a controller that handles all requests from a server, which is useful to achieve flexibility and reuse code without redundancy.

 
\begin{figure}[!htp]
    \centering
    \includegraphics[width=0.65\textwidth]{img/architecture/Front_Controller.png}
    \caption{Example of Front Controller architecture.}
    \label{fig:front-controller}
\end{figure}

These front controllers are normally used in web applications like the one in this project to implement flows. It serves to simplify navigation between screens and is invoked every time calls are made to the server. This Front Controller handles all the tasks common to the application, such as session, caching and in our case routing and invocation of data extraction through implemented methods.

This was the most interesting option and the one that best suited what we were looking for in this Master Thesis, as the project took into account the features that we needed and the simplicity the model provides to perform and implement those features. 

The features that we required from this sever are as mentioned before, three main tasks, which are data extraction, data processing with its prediction, and data serving through the API which we will discuss in the following.

% 1. Extraccion de datos
\myparagraph{Data extraction}
The focus of this project is on the social network Twitter and because of this, it is needed to extract information from Twitter in order to achieve the goal. The project wants to analyse the suffering per user of an Eating Disorder so it is needed to extract the latest tweets of a specific person in order to treat them and introduce them as input to our model.

For this, Python-based tool was used called snscrape~\cite{JustAnot83:online}, which can extract data from the Twitter API without us having to authenticate ourselves. 

This snscrape Python package that can be easily installed using the pip manager. It allows to take as many tweets as desired from a profile, and put them in a text file in JSON format. An example of this can be seen in the code block described in Appendix~\ref{appendix:JSONsnscrape}.

So, it was decided to use snscrape for its simplicity and for not needing to authenticate to the API. With the command written in the Code~\ref{code:snscrape}, the 1000 most recent tweets of the user with the identifier \$username can be accessed for then being saved in JSON format in the file user-tweets.json.

Subsequently, this file will be read to be able to process the information taken. The tool has more functionalities such as searching Twitter (twitter-search) or consulting a profile (twitter-profile) but this one is enough for the given approach.

\begin{lstlisting}[caption={Command executed for obtaining tweets.}, label={code:snscrape}]
snscrape --jsonl --max-results 1000 twitter-user $username > user-tweets.json
\end{lstlisting}

Once the file is ready, it is read using the pandas library in order to ingest it into the model, preprocessing it beforehand. This file is given a temporal character because after reading the file, it is deleted from the server so as not to generate overload. Now, in the next paragraph, we will explain how we preprocess the text once it is read.

% 2. Procesamiento de datos (empaquetamiento de modelo) / wordcloud y prediccion
\myparagraph{Data processing and prediction}
Data obtained from twitter in our platform is needed to be processed. The information will be treated in a similar way in this Flask server to the one explained in~\ref{sec:preprocessing}.

So, what the project does is reusing the code written in Code~\ref{code:preprocessing} for this data extracted from Twitter. As a result an array of words is obtained from the last tweets published by the user in question.

%Wordcloud
In addition, the project took the opportunity of having this to create more statistics and to give the user more insights. For example, the creation of a wordcloud has been implemented. It reflects the most used words and their frequency in a very intuitive graphical way.

% 3. Como se sirven los datos: rutas y por JSON --> API
\myparagraph{Data serving through API}
%JSON Format and wordcloud img processing
To serve the prediction and additional statistics the API has been programmed. This API has a dedicated path to return the extracted results in JSON format.

The simplified output obtained by querying the API as can be seen in the Code~\ref{code:jsonapi} is a composition of the prediction result, user information and the wordcloud image in base64 format. This information is then used in the Flutter part to build the user interface.

%% METER EJEMPLO JSON
\begin{lstlisting}[caption={JSON output from API}, label={code:jsonapi}]
{
    "prediction": "0", 
    "username": "elonmusk", 
    "pic": "https://pbs.twimg.com/profile_images/1529956155937759233/Nyn1HZWF.jpg", 
    "wordcloud": "iVBORw0KGgoAAAANSUhEUgAAAZAAAADICAIAAAB..."
}
\end{lstlisting}


In order to be able to send the wordcloud image through the API, it has to be encoded in base64 format. Subsequently, it has to be decoded in the application part in order to be displayed.\\

Besides these mentioned tasks that are directly related to the web server functioning, the module has been also equipped with a Semantic Events Trigger. This block is throwing an event against the Task Automation Server in order to trigger actions that are reflected on the user interface. We will talk about this integration in the Section~\ref{sec:taskautomation}.

\section{Task Automation}
\label{sec:taskautomation}
As mentioned before, our system has been integrated with a semantic task automation platform in order to enhance the functionalities of our platform and also the user experience. The project has used a Task Automation Service (TAS) hosted by GSI department of the Universidad Politécnica de Madrid which provides the ability of online automation.

The system will be based on Evented WEb (EWE) as indicated in the Chapter~\ref{chap:state-of-art}. This ontology has four major classes on its core which are Channel, Event, Action and Rule, and it is needed to have a semantic representation for putting an automation in place. In this Section we will detail the representations developed for this Master Thesis

Before that, for setting the ground, the architecture on which this semantic task automation is constructed is presented. As discussed it is a service which is running online which is constructed as the Figure~\ref{fig:ewearchitecure} is showing.

\begin{figure}[h]
    \centering
    \includegraphics[width=0.85\textwidth]{img/architecture/EWEarchitecture.png}
    \caption{EWE architecture for project's system.}
    \label{fig:ewearchitecure}
\end{figure}

As it can be seen this Automation Server is connected both to our user interface and the web server. When a positive prediction is performed in the Flask server, an event is triggered, arriving to the Events Manager of the Task Automation Server. Once the event is received, it interacts with the Rule Engine which decides what to do with that event depending on the configured rules. 

The engine is responsible for evaluating the received events along with the stored rules, in order to trigger the corresponding actions. Besides, Rule Engine is connected with the Channel and the Rule Administrators, which provide the necessary resources for dispatching the rule. These are modules that have the same purpose but for different objects supported by the ontology. Both have a repository where the objects are stored and also they are provided with a manager and an editor, which serve for modifying the objects or even delete them.

When the rule is processed in the Rule Engine, it is contacting the Actions Trigger which activates the action that is indicated. This Actions Trigger is connected directly to the Flutter Interface, unleashing the necessary process that are desired for the user.

% ARCHITECTURE: pillar de aqui https://books.google.es/books?hl=es&lr=&id=HhKhDQAAQBAJ&oi=fnd&pg=PA33&ots=1Q4idJtH39&sig=6-uS1BX6NGtWcwug1GOMoIlns1s&redir_esc=y#v=onepage&q&f=false

A semantic model of our system was needed in order to automatise the actions the project wants to offer. These actions recommend the user to talk to friends, contact an organisation or make an appointment with a specialized doctor. These actions are triggered when a positive response is made.


% Meter representacion creada
This created and custom representation with the declaration of channel, events and actions can be seen in the following Code~\ref{code:ewerepresentation}.

\begin{lstlisting}[label={code:ewerepresentation}, caption={Custom semantic representation implemented.}]
###################################
# Channel definition
###################################
ewe-ed:EDPrediction a owl:Class ;    
    rdfs:label "Prediction of Eating Disorder made by platform"@en ;
    rdfs:comment "This channel represents the prediction emited by the system"@en
    rdfs:subClassOf ewe:Channel .

###################################
# Events definition
###################################

ewe-ed:EDPredictionPositive a owl:Class ;
    rdfs:label "User has been predicted as positive in Eating Disorder"@en ;
    rdfs:comment "This Trigger fires when a prediction is positive in Eating Disorder."@en ;
    rdfs:subclassOf ewe:Event ;
    rdfs:domain ewe-ed:EDPrediction .

ewe-ed:EDPredictionNegative a owl:Class ;
    rdfs:label "User has been predicted as negative in Eating Disorder"@en ;
    rdfs:comment "This Trigger fires when a prediction is negative in Eating Disorder."@en ;
    rdfs:subclassOf ewe:Event ;
    rdfs:domain ewe-ed:EDPrediction .

###################################
# Actions definition
###################################
ewe-ed:RecommendFriends a owl:Class ;
    rdfs:label "Recommend Friends"@en ;
    rdfs:comment "This action will tell to call a friend."@en ;
    rdfs:subclassOf ewe:Action ;
    rdfs:domain ewe-ed:EDPrediction .

ewe-ed:RecommendNGO a owl:Class ;
    rdfs:label "Recommend NGO"@en ;
    rdfs:comment "This action will tell to contact an organisation"@en ;
    rdfs:subclassOf ewe:Action ;
    rdfs:domain ewe-ed:EDPrediction .

ewe-ed:RecommendDoctor a owl:Class ;
    rdfs:label "Recommend Doctor"@en ;
    rdfs:comment "This action will tell to contact an expert"@en ;
    rdfs:subclassOf ewe:Action ;
    rdfs:domain ewe-ed:EDPrediction .

\end{lstlisting}

Besides this, a set of rules has been implemented. These will trigger the actions that we have implemented there when a prediction is positive. All of this will be charged into the Task Automation Server so it can understand and talk theon layer of this M system's language.

Examples of these rules can be as following:
\begin{itemize}
    \item ewe-ed:EDPredictionPositive $\rightarrow$ ewe-ed:RecommendFriends
    \item ewe-ed:EDPredictionPositive $\rightarrow$ ewe-ed:RecommendNGO
    \item ewe-ed:EDPredictionPositive $\rightarrow$ ewe-ed:RecommendDoctor
    
\end{itemize}

