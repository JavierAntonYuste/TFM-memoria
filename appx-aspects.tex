\chapter{Social, economic, environmental and ethical aspects}

In this section we will analyse the social, economic, environmental an ethical impact of this Master Thesis.

\section{Introduction}
Today's technologies allow universal access to research resources, so this work can be consulted anywhere in the world as it can be found on the department's website, having a global impact.

The development of this project provides a contribution that not only affects the area of research and in this particular case research in Telecommunications. It also affects the area of health in which this project is framed, as it has been developed and will be exposed in a public way.


\section*{Social impact}
This work focuses on the health framework, specifically on mental health and Eating Disorders, affecting people related to this topic. This project intends to provide help for the early detection of these pathologies in order to accelerate the diagnostic process without the intention of substituting anything already established.

With this, it is intended to reinforce healthcare in early detection, as it is the most key and critical step in the process. The patient's recovery time and the extent of the disease depends on optimising it.

\section*{Economic impact}
Regarding the economic impact, this project doesn't have a huge economic implication other than the cost it could save in the long run if the disease is treated before and takes less recovering time for the patient. The distribution is thought to be open and public so there won't be any direct economical repercussion.

\section*{Environmental impact}

Considering the nature of this project, it can be assured that it does not have a notable environmental impact, as the only resources used are the electronic equipment and the electricity that supplies it, these resources being very limited.

\section*{Ethical and professional responsibility}

This Master Thesis tackles information which is really sensible and requires a lot of awareness and consciousness on what is been treated. This is why we have left the system with disclaimers and comments that this is not a scientific-proven platform rather a support and helper for the medical system.

\section*{Conclusions}

This project will help potential users suffering from an Eating Disorder to consult a prediction of their approximate profile. They are encouraged to take the next step to be treated if they are positive, in order to improve and treat the illness appropriately with a dedicated organisation or a medical specialist. It is the germ of a great concept, as it has great potential, which is expected to be exploited in the near future.
