\cleardoublepage
\phantomsection
\chapter*{Abstract}
\addcontentsline{toc}{chapter}{Abstract}

Mental illnesses and specifically Eating Disorders are latent problems in today's society that need to be focused on for improving their situation. In recent years, the state of people's mental health has been affected by different causes. One of the most important of these causes is the COVID-19 pandemic. This has led to the development or aggravation of these diseases in patients who did not suffer from them before.

It is therefore important to detect and treat them as early as possible. Even more important is to reduce the time between the onset of the first symptoms until they are identified and assigned to an effective treatment. Medical specialists are dared with real challenges when confronted with these diseases, as mental health is one of the areas that has historically received the least social and economic focus. This project aims to make a contribution to help in this task of raising awareness and early detection of Eating Disorders.

Because of this, a system for the recognition of Eating Disorders based on machine learning and natural language processing technologies has been developed. The main objective of the system is to predict, through the interactions that users have on social networks, specifically Twitter, the potential suffering of these problems.

In order to be able to reach the user, the system has been equipped with an easy and user-friendly user interface together with a web server that helps to process the posts of the analysed profile. The system has also been integrated with a task automation platform that enables to advance the regulation if the prediction is positive.

Looking at the potential that this automatic learning can have in the field of health, systems such as the one presented in this Master's Thesis can be a very promising solution in the future and improve the quality of life of a person suffering from these Eating Disorders.
\vfill
\textbf{Keywords:} Eating Disorders, Anorexia, Bulimia, Binge Eating Disorder, Automatic Learning, early detection, Flutter, Flask, TAS.