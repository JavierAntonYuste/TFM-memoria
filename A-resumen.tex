\cleardoublepage
\phantomsection
\chapter*{Resumen}
\addcontentsline{toc}{chapter}{Resumen}

Las enfermedades mentales y en concreto los Trastornos de Conducta Alimentaria son problemas latentes de la sociedad actual en las que se necesita poner foco para poder mejorar su situación. Durante estos últimos años, el estado de salud mental de las personas se ha visto afectado por diferentes causas, entre ellas la más importante ha sido la pandemia de COVID-19. Esta ha propiciado el desarrollo o el agravamiento de estas enfermedades en pacientes que antes no las sufrían.

Es por ello por lo que es importante detectarlas y tratarlas con la mayor rapidez posible. Más importante aún es reducir el tiempo que trascurre entre que empiezan los primeros síntomas hasta que son identificadas y asignadas un tratamiento efectivo. Los médicos especialistas son desafiados con verdaderos retos cuando se enfrentan a estas enfermedades, ya que la salud mental es una de las áreas en las que se ha puesto menos enfoque históricamente a nivel social y económico. Este proyecto pretende hacer una contribución para poder ayudar en esta tarea de aumento de la concienciación y la detección temprana de Trastornos de Conducta Alimentaria.

Debido a esto, se ha desarrollado un sistema de reconocimiento del padecimiento de Trastornos de Conducta Alimentaria basado en tecnologías de aprendizaje automático y procesado de lenguaje natural. El objetivo principal del mismo es el de predecir a través de las interacciones que los usuarios tienen en redes sociales, en concreto Twitter, el potencial padecimiento de estos problemas.

Para poder hacerlo llegar al usuario, se ha equipado al sistema con una interfaz de usuario fácil y sencilla de usar junto con un servidor web que ayuda a procesar las publicaciones del perfil analizado. Se ha integrado el sistema además con una plataforma de automatización de tareas que da pie a avanzar en el proceso de regulación la enfermedad en el caso de ser nuestra predicción positiva.

Mirando el potencial que el aprendizaje automático puede tener en el ámbito de la salud, sistemas como el presentado en este Trabajo de Fin de Máster pueden suponer una solución muy beneficiosa en el futuro y mejorar la calidad de vida de una persona que padece estos Trastornos de Conducta Alimentaria.
\vfill
\textbf{Palabras clave:} Trastornos de Conducta Alimentaria, Anorexia, Bulimia, Atracones, Aprendizaje Automático, detección temprana, Flutter, Flask, TAS.